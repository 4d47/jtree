\documentclass[journal]{IEEEtran}
\usepackage{blindtext}
\usepackage{graphicx}
\usepackage{listings}
\usepackage[superscript,biblabel]{cite}


\hyphenation{op-tical net-works semi-conduc-tor}

\begin{document}

\title{A Grammar Notation for ETNs}

\author{Breck~Yunits% <-this % stops a space
\thanks{Breck Yunits is a researcher at Ohayo Computer (breck@ohayo.computer)}% <-this % stops a space
}

\markboth{September~2017}%
{Shell \MakeLowercase{\textit{et al.}}: Bare Demo of IEEEtran.cls for Journals}

\maketitle


\begin{abstract}
%\boldmath
I introduce the core idea of a new grammar notation for formally describing programming languages that extend Tree Notation.
\end{abstract}

\IEEEpeerreviewmaketitle

\section{Introduction}

Creating a great programming language is a multi-step process. One step in that process is to define a language in a grammar notation such as BNF. Unfortunately, like the programming languages they describe, these grammar notations are complex and error-prone.

Below I introduce the core idea of a much simpler grammar notation for defining programming languages that Extend Tree Notation (ETNs).

\section{A Notation for ETN Grammars}

An ETN Grammar is a \textit{single} consisting of a set of Node Type Definitions.

A Node Type Definition is a \textit{double} consisting of a unique node type identifier and an ETN Grammar.

Everything is encoded in Tree Notation, hence the grammar notation itself is an ETN.

\section{Example}

An ETN Grammar file for an imagined ETN called Tally, with 2 possible recursive node types \{+, -\} might look like this:

\begin{lstlisting}
Tally
 + Tally
 - Tally
\end{lstlisting}

A valid program in the Tally language defined by the file above:

\begin{lstlisting}
+ 4 5
 - 1 1
\end{lstlisting}

\section{Conclusion and Future Work}

The introduction above is minimal but shows the core idea: ETNs can be formally defined in a simple grammar notation that itself is an ETN.

Ohayo Computer has developed a feature-rich compiler compiler for these grammar files. Future publications and/or open source releases will delve into the additional features found in the compiler compiler and its associated ETN.

\end{document}
